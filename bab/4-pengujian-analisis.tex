\chapter{HASIL DAN PEMBAHASAN}
\label{chap:hasilpembahasan}

% Ubah bagian-bagian berikut dengan isi dari hasil dan pembahasan

Pada penelitian ini dipaparkan hasil pengujian dari desain sistem dan implementasi yang telah dilakukan. Tahap pengujian dibagi menjadi beberapa bagian, yaitu:

\begin{enumerate}
	\item Pengujian pengaruh batch size dan epoch terhadap model
	\begin{enumerate}
		\item Batch size = 8; Epoch = 80
		\item Batch size = 8; Epoch = 100
		\item Batch size = 16; Epoch = 80
		\item Batch size = 16; Epoch = 100
		\item Batch size = 32; Epoch = 80
		\item Batch size = 32; Epoch = 100
	\end{enumerate}
	
	\item Pengujian pengaruh durasi input audio terhadap hasil klasifikasi
	\begin{enumerate}
		\item 30 detik
		\item 15 detik
		\item 9 detik
		\item 3 detik
	\end{enumerate}
	
\end{enumerate}

% Skenario pengujian berupa hasil penelitian dan perancangan yang telah dibuat pada bab 3

\section{Skenario Pengujian}
\label{sec:skenariopengujian}

Pengujian dilakukan dengan \lipsum[1-2]

\section{Evaluasi Pengujian}
\label{sec:analisispengujian}

Dari pengujian yang \lipsum[1]

% Contoh pembuatan tabel
\begin{longtable}{|c|c|c|}
  \caption{Hasil Pengukuran Energi dan Kecepatan}
  \label{tb:EnergiKecepatan}\\
  \hline
  \rowcolor[HTML]{C0C0C0}
  \textbf{Energi} & \textbf{Jarak Tempuh} & \textbf{Kecepatan} \\
  \hline
  10 J & 1000 M & 200 M/s \\
  20 J & 2000 M & 400 M/s \\
  30 J & 4000 M & 800 M/s \\
  40 J & 8000 M & 1600 M/s \\
  \hline
\end{longtable}

\lipsum[2-4]
