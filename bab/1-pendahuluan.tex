\chapter{PENDAHULUAN}
\label{chap:pendahuluan}

% Ubah bagian-bagian berikut dengan isi dari pendahuluan

\section{Latar Belakang}
\label{sec:latarbelakang}

Genre musik adalah metode pengkategorian untuk mengidentifikasikan suatu musik ke dalam suatu tradisi yang sudah pernah ada \citep{Genre}. Musik dapat dibedakan menjadi berbagai macam genre, seperti Pop, Rock, Blues, Jazz, Religi, dan lain sebagainya. Sifat dari musik sendiri adalah subjektif sehingga klasifikasi genre tiap orang bisa jadi berbeda-beda dan mungkin juga ada beberapa genre yang saling tumpang tindih satu sama lain.

Penyebaran penggunaan internet telah membawakan perubahan yang besar pada dunia industri permusikan. Perubahan-perubahannya adalah seperti menyebarnya platform untuk mendengarkan musik secara daring, serta klasifikasi genre musik yang kemudian digunakan sebagai sistem rekomendasi. Dewasa ini, dengan perkembangan platform untuk mendengarkan musik secara daring, pengguna dapat mendengarkan musik yang mereka sukai dimana saja dan kapan saja, serta mereka dapat mencari jutaan musik melalui platform-platform tersebut, seperti Spotify, Bandcamp, Deezer, SoundCloud, dan lain sebagainya \citep{Elbir2018MusicGC}.

Pada hari Senin, 22 Februari 2021, Spotify telah mengkonfirmasi bahwa lebih dari 60.000 trek baru diunggah ke platformnya setiap hari. Menurut Jeremy Elrich, co-Head of Music dari Spotify, hal ini berarti pada akhir tahun ini dapat diperkirakan terdapat 22 juta trek musik yang akan ditambahkan ke katalog musik Spotify. Dengan angka 60.000 per hari, dapat dihitung bahwa setiap 1,4 detik terdapat satu trek baru yang diunggah. Spotify mengkonfirmasi bahwa pada bulan November tahun lalu, platform tersebut memiliki sekitar 70 juta trek yang tersimpan di dalamnya, dari sini maka dapat diperkirakan bahwa pada akhir tahun ini Spotify akan menjadi rumah untuk lebih dari 90 juta trek musik, dan akan menjadi lebih dari 100 juta pada awal tahun berikutnya \citep{musicb}.

\section{Permasalahan}
\label{sec:permasalahan}

Dengan semakin banyaknya musik yang diunggah tiap harinya, maka sangatlah tidak mungkin bagi manusia untuk mengkategorikan musik yang telah diunggah ke genre yang sesuai secara manual. Oleh karena itu, diperlukan deep learning untuk klasifikasi genre musik agar proses pengkategorian musik dapat dilakukan dengan lebih cepat dan akurat.

\section{Batasan Masalah}
\label{sec:batasanmasalah}

Batasan-batasan permasalahan dari penelitian ini adalah sebagai berikut:

\begin{enumerate}[nolistsep]

  \item Dataset yang digunakan untuk training dan testing adalah Free Music Archive (FMA) Dataset yang berisikan data trek musik dengan durasi 30 detik.

  \item Total 3 genre dengan masing-masing 3 subgenre yang akan diklasifikasikan.

  \item Mel-frequency cepstral coefficients (MFCC) digunakan sebagai fitur audio untuk proses training.
  
  \item Training dilakukan dengan menggunakan Convolutional Neural Network (CNN).
  
  \item Hasil klasifikasi menampilkan 3 subgenre dengan probabilitas tertinggi.
 
\end{enumerate}

\section{Tujuan}
\label{sec:Tujuan}

Penelitian ini bertujuan untuk membuat suatu sistem yang dapat mengklasifikasikan beberapa subgenre yang paling masuk akal dari suatu input data musik yang diambil dari genre yang berbeda-beda serta dengan durasi yang relatif fleksibel. Sistem klasifikasi ini dilakukan dengan menggunakan Mel-frequency cepstral coefficients (MFCC) sebagai fitur utama serta Convolutional Neural Network (CNN) sebagai framework deep learning-nya.

\section{Manfaat}
\label{sec:manfaat}

Manfaat dari penelitian ini adalah untuk mempermudah pengklasifikasian subgenre trek musik dari genre yang berbeda-beda yang berjumlah banyak dan memiliki durasi yang terbatas.

% Format Buku TA baru, ga pake sistematika penulisan

% \section{Sistematika Penulisan}
% \label{sec:sistematikapenulisan}

% Laporan penelitian tugas akhir ini terbagi menjadi \lipsum[1][1-3] yaitu:

% \begin{enumerate}[nolistsep]

%   \item \textbf{BAB I Pendahuluan}

%   Bab ini berisi \lipsum[2][1-5]

%   \vspace{2ex}

%   \item \textbf{BAB II Tinjauan Pustaka}

%   Bab ini berisi \lipsum[3][1-5]

%   \vspace{2ex}

%   \item \textbf{BAB III Desain dan Implementasi Sistem}

%   Bab ini berisi \lipsum[4][1-5]

%   \vspace{2ex}

%   \item \textbf{BAB IV Pengujian dan Analisa}

%   Bab ini berisi \lipsum[5][1-5]

%   \vspace{2ex}

%   \item \textbf{BAB V Penutup}

%   Bab ini berisi \lipsum[6][1-5]

% \end{enumerate}
