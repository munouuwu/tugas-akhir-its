\begin{center}
  \large\textbf{ABSTRACT}
\end{center}

\addcontentsline{toc}{chapter}{ABSTRACT}

\vspace{2ex}

\begingroup
  % Menghilangkan padding
  \setlength{\tabcolsep}{0pt}

  \noindent
  \begin{tabularx}{\textwidth}{l >{\centering}m{3em} X}
    % Ubah kalimat berikut dengan nama mahasiswa
    \emph{Name}     &:& Dimas Iqbal Fahreza \\

    % Ubah kalimat berikut dengan judul tugas akhir dalam Bahasa Inggris
    \emph{Title}    &:& \emph{Music Genre And Subgenre Classification Based on Mel-frequency Cepstral Coefficients Features Using Convolutional Neural Network} \\

    % Ubah kalimat-kalimat berikut dengan nama-nama dosen pembimbing
    \emph{Advisors} &:& 1. Reza Fuad Rachmadi, S.T., M.T., Ph.D \\
                    & & 2. Dr. Eko Mulyanto Yuniarno, S.T., M.T. \\
  \end{tabularx}
\endgroup

% Ubah paragraf berikut dengan abstrak dari tugas akhir dalam Bahasa Inggris
\emph{Within the increase of music uploaded to the internet, also the development of online music platforms such as Spotify, Bandcamp, Deezer, SoundCloud, etc. as of lately, users can enjoy their favorite music and artists anytime anywhere. Such amount of musics have to be classified to their own genres and subgenres automatically because it's impossible for humans to do such thing manually. In that case, it's important to create a system that can classify genre and subgenre of music with more efficiency and accuracy. This research will test whether Convolutional Neural Network (CNN) algorithm can be used to classify music input to its top 3 subgenres using Mel-Frequency Coefficient of Cepstrum (MFCC) as its main feature.}

% Ubah kata-kata berikut dengan kata kunci dari tugas akhir dalam Bahasa Inggris
\emph{Keywords}:
\emph{Classification}, \emph{Music Subgenre}, \emph{Mel-Frequency Coefficient of Cepstrum}, \emph{Convolutional Neural Network}.
