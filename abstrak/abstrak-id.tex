\begin{center}
  \large\textbf{ABSTRAK}
\end{center}

\addcontentsline{toc}{chapter}{ABSTRAK}

\vspace{2ex}

\begingroup
  % Menghilangkan padding
  \setlength{\tabcolsep}{0pt}

  \noindent
  \begin{tabularx}{\textwidth}{l >{\centering}m{2em} X}
    % Ubah kalimat berikut dengan nama mahasiswa
    Nama Mahasiswa    &:& Dimas Iqbal Fahreza \\

    % Ubah kalimat berikut dengan judul tugas akhir
    Judul Tugas Akhir &:&	Klasifikasi Genre Dan Subgenre Musik Berbasis Fitur Mel-frequency Cepstral Coefficients Menggunakan Convolutional Neural Network \\

    % Ubah kalimat-kalimat berikut dengan nama-nama dosen pembimbing
    Pembimbing        &:& 1. Reza Fuad Rachmadi, S.T., M.T., Ph.D \\
                      & & 2. Dr. Eko Mulyanto Yuniarno, S.T., M.T. \\
  \end{tabularx}
\endgroup

% Ubah paragraf berikut dengan abstrak dari tugas akhir
Dewasa ini, dengan semakin banyaknya musik yang diunggah ke internet, serta 
perkembangan platform untuk mendengarkan musik secara daring, seperti Spotify, 
Bandcamp, Deezer, SoundCloud, dan lain sebagainya, pengguna dapat mendengarkan musik 
yang mereka sukai dimana saja dan kapan saja. Jumlah musik yang sangat banyak tersebut 
perlu untuk dikategorikan berdasarkan genre dan subgenrenya secara otomatis karena sangatlah tidak 
mungkin bagi manusia untuk melakukannya secara manual. Oleh karena itu, diperlukan 
suatu sistem klasifikasi genre dan subgenre musik berbasis deep learning agar pengkategorian genre dan subgenre musik dapat dilakukan dengan lebih efisien dan akurat. Penelitian ini menguji apabila algoritma Convolutional Neural Network (CNN) dapat digunakan untuk mengklasifikasikan subgenre musik dengan cara menampilkan tiga subgenre teratas dengan menggunakan Mel-Frequency Coefficient of Cepstrum (MFCC) sebagai fitur utamanya.

% Ubah kata-kata berikut dengan kata kunci dari tugas akhir
Kata Kunci: Klasifikasi, Subgenre Musik, Mel-Frequency Coefficient of Cepstrum, Convolutional Neural Network
