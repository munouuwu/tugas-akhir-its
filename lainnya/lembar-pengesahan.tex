\begin{center}
	\large
  \textbf{LEMBAR PENGESAHAN}
\end{center}

% Menyembunyikan nomor halaman
\thispagestyle{empty}

\begin{center}
  % Ubah kalimat berikut dengan judul tugas akhir
  \textbf{KLASIFIKASI GENRE DAN SUBGENRE MUSIK BERBASIS FITUR MEL-FREQUENCY CEPSTRAL COEFFICIENTS MENGGUNAKAN CONVOLUTIONAL NEURAL NETWORK}
\end{center}

\begingroup
  % Pemilihan font ukuran small
  \small
  
  \vspace{3ex}

  \begin{center}
    \textbf{PROPOSAL TUGAS AKHIR}
    \\Diajukan untuk memenuhi salah satu syarat
    \\memperoleh gelar Sarjana Teknik pada
    \\Program Studi S-1 Teknik Komputer
    \\Departemen Teknik Komputer
    \\Fakultas Teknologi Elektro dan Informatika Cerdas
    \\Institut Teknologi Sepuluh Nopember
  \end{center}

  \vspace{3ex}

  \begin{center}
    % Ubah kalimat berikut dengan nama dan NRP mahasiswa
    Oleh: Dimas Iqbal Fahreza
    \\NRP. 0721 18 4000 0045
  \end{center}

  \vspace{3ex}

  % \begin{center}
  % Ubah kalimat-kalimat berikut dengan tanggal ujian dan periode wisuda
  %   Tanggal Ujian : 1 Juni 2021\\
  %   Periode Wisuda : September 2021
  % \end{center}

  \begin{center}
    Disetujui oleh Tim Penguji Tugas Akhir:
  \end{center}

  \vspace{4ex}

  \begingroup
    % Menghilangkan padding
    \setlength{\tabcolsep}{0pt}

    \noindent
    \begin{tabularx}{\textwidth}{X l}
      % Ubah kalimat-kalimat berikut dengan nama dosen pembimbing pertama
      1. Reza Fuad Rachmadi, S.T., M.T., Ph.D.          & Pembimbing \\
      % NIP: 18560710 194301 1 001        & ................................... \\
      &  \\
      &  \\
      % Ubah kalimat-kalimat berikut dengan nama dosen pembimbing kedua
      2. Dr. Eko Mulyanto Yuniarno, S.T., M.T.     & Ko-Pembimbing \\
      &  \\
      &  \\
      % Ubah kalimat-kalimat berikut dengan nama dosen penguji pertama
      3. Dr. Galileo Galilei, S.T., M.Sc.  & Penguji I \\
      &  \\
      &  \\
      % Ubah kalimat-kalimat berikut dengan nama dosen penguji kedua
      4. Friedrich Nietzsche, S.T., M.Sc.  & Penguji II \\
      &  \\
      &  \\
      % Ubah kalimat-kalimat berikut dengan nama dosen penguji ketiga
      5. Alan Turing, ST., MT.             & Penguji III \\
      &  \\
      &  \\
    \end{tabularx}
  \endgroup

  \vspace{12ex}

  % \begin{center}
  %   % Ubah kalimat berikut dengan jabatan kepala departemen
  %   Mengetahui, \\
  %   Kepala Departemen Teknik Dirgantara FTD - ITS \\

  %   \vspace{8ex}

  %   % Ubah kalimat-kalimat berikut dengan nama dan NIP kepala departemen
  %   \underline{Dr. Leonardo Da Vinci, S.T., M.T.} \\
  %   NIP. 14520415 151905 1 001
  % \end{center}

  \begin{center}
    \textbf{SURABAYA\\Mei, 2022}
  \end{center}
\endgroup
