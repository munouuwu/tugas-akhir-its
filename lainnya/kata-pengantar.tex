\begin{center}
  \Large
  \textbf{KATA PENGANTAR}
\end{center}

\addcontentsline{toc}{chapter}{KATA PENGANTAR}

\vspace{2ex}

% Ubah paragraf-paragraf berikut dengan isi dari kata pengantar

Puji dan syukur kehadirat Tuhan Yang Maha Esa atas segala karunia-Nya, penulis dapat menyelesaikan penelitian ini yang berjudul \textbf{Klasifikasi Genre Dan Subgenre Musik Berbasis Fitur Mel-frequency Cepstral Coefficients Menggunakan Convolutional Neural Network (\emph{Music Genre And Subgenre Classification Based on Mel-frequency Cepstral Coefficients Features Using Convolutional Neural Network})}.

 Penelitian ini disusun dalam rangka memenuhi bidang riset di Departemen Teknik Komputer ITS serta digunakan sebagai persyaratan menyelesaikan pendidikan S1. Oleh karena itu, penulis mengucapkan terima kasih kepada:

\begin{enumerate}[nolistsep]

  \item Keluarga, Ibu, Bapak dan Saudara tercinta yang telah memberikan segala bentuk dukungan

  \item Bapak Reza Fuad Rachmadi, S.T., M.T., selaku dosen pembimbing 1

   \item Bapak Dr. Eko Mulyanto Yuniarno, S.T., M.T., selaku dosen pembimbing 2
   
   \item Bapak-ibu dosen pengajar Departemen Teknik Komputer ITS atas segala ilmu dan arahannya
   
   \item Teman-teman Asisten Lab B201 yang telah menemani dan mendukung
   
   \item Serta teman-teman Teknik Komputer angkatan 2018 yang telah bersama-sama menempuh segala kegiatan perkuliahan dari awal sampai akhir masa kuliah penulis

\end{enumerate}

Kesempurnaan hanya milik Tuhan Yang Maha Esa, untuk itu
penulis memohon segenap kritik dan saran yang membangun. Semoga penelitian ini dapat memberikan manfaat bagi kita semua.
Amin.

\begin{flushright}
  \begin{tabular}[b]{c}
    % Ubah kalimat berikut dengan tempat, bulan, dan tahun penulisan
    Surabaya, Mei 2022\\
    \\
    \\
    \\
    \\
    % Ubah kalimat berikut dengan nama mahasiswa
    Penulis
  \end{tabular}
\end{flushright}
